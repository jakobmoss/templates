%% ============================================================================
%%
%%  Studieretningsprojekt (SRP)
%%
%%  Forfatter: FORNAVN EFTERNAVN
%%
%%  Preamble
%%
%%  Original preamble: Jakob Lysgaard Rørsted (Mosumgaard)
%% ============================================================================

% ~~~~~~~~~~~~~~~~~~~~~~~~~~~~~~~~~~~~~~~~~~~~~~~~~~~~~~~~~~~~~~~~~~~~~~~~~~~~~
% Core
% ~~~~~~~~~~~~~~~~~~~~~~~~~~~~~~~~~~~~~~~~~~~~~~~~~~~~~~~~~~~~~~~~~~~~~~~~~~~~~
% Essentielle pakker
\usepackage[utf8]{inputenc}
\usepackage[T1]{fontenc}

% Microtype -- Gør alting pænere
\usepackage{microtype}
\microtypesetup{final}

% Diverse værktøjer
\usepackage{etoolbox}

% Inklusion af kode, hvis det ønskes
% --> Needs to be loaded this early to avoid problems with some font packages
% NOTE: This packages requires the Python-module 'Pygments' to be installed
% NOTE: Fails unless this file is compiled with '--shell-escape' !
% \usepackage{minted}
% \usemintedstyle{friendly}

% Skift sprog og orddeling til dansk
\usepackage[danish]{babel}
\renewcommand\danishhyphenmins{22}  % Til danske "hv-ord".

% DANISH STUFF
% Navn på indholdsfortegnelsen
\addto\captionsdanish{\renewcommand{\contentsname}{Indholdsfortegnelse}}

% Fonts: Linux Libertine (http://www.tug.dk/FontCatalogue/linuxlibertine/)
\usepackage{libertine}            % Linux Libertine som text font
\usepackage{libertinust1math}     % Matematik til  Linux Libertine
\usepackage[scaled=.95]{newtxtt}  % Pæn teletype
\linespread{1.1}                  % For at antallet af tegn passer


% ~~~~~~~~~~~~~~~~~~~~~~~~~~~~~~~~~~~~~~~~~~~~~~~~~~~~~~~~~~~~~~~~~~~~~~~~~~~~~
% Sideopsætning
% ~~~~~~~~~~~~~~~~~~~~~~~~~~~~~~~~~~~~~~~~~~~~~~~~~~~~~~~~~~~~~~~~~~~~~~~~~~~~~
% Marginer (for at antallet af karakterer passer)
\setlrmarginsandblock{4.0cm}{*}{1}
\setulmarginsandblock{*}{4.7cm}{0.8}

% Layout tricks (fjern ikke!)
\setlength{\topskip}{1.6\topskip}
\checkandfixthelayout
\sloppybottom
\strictpagechecktrue


% ~~~~~~~~~~~~~~~~~~~~~~~~~~~~~~~~~~~~~~~~~~~~~~~~~~~~~~~~~~~~~~~~~~~~~~~~~~~~~
% Page and chapter styles
% ~~~~~~~~~~~~~~~~~~~~~~~~~~~~~~~~~~~~~~~~~~~~~~~~~~~~~~~~~~~~~~~~~~~~~~~~~~~~~

% Genvej til nogle skriftstørrelser
\newcommand{\foliofont}{\color{black}\sffamily}
\newcommand{\headerfont}{\color{black}\sffamily}

% Lav en new og simpel pagestyle (kun odd da enkeltsidet)
\makepagestyle{jlrsrp}
\makeoddhead {jlrsrp}{\headerfont\theauthor} {} {\headerfont\projecttitle}
\makeoddfoot {jlrsrp}{}{\foliofont\thepage\ / \thelastpage}{}

% Skriv afsnit med en ikke-serif font
\setsecheadstyle{\Large\bfseries\sffamily\raggedright}
\setsubsecheadstyle{\large\bfseries\sffamily\raggedright}

% Sørg for at kapiterlsiderne opfører sig ordentligt (even er redundant)
\copypagestyle{chapter}{empty}
\makeoddfoot{chapter}{}{\foliofont\thepage}{}
\makeevenfoot{chapter}{}{\foliofont\thepage}{}

% Pæne kapitler
\usepackage{graphicx}
\makechapterstyle{jrmbsc}{% requires graphicx package
  \chapterstyle{default}
  \renewcommand*{\chapnamefont}{%
    \normalfont\LARGE\color{black}\scshape\raggedleft}
  \renewcommand*{\chaptitlefont}{%
    \normalfont\Huge\color{black}\sffamily\raggedleft}
  \renewcommand*{\chapternamenum}{}
  \renewcommand*{\printchapternum}{%
    \makebox[0pt][l]{\hspace{0.4em}
      \resizebox{!}{5ex}{%
        \normalfont\Large\color{black}\thechapter}
    }%
  }%
  \renewcommand*{\afterchapternum}{%
    \par\hspace{1.5cm}\color{black}\hrule\vskip\midchapskip}}

% Aktiver pagestyle og chapterstyle
\pagestyle{jlrsrp}
\chapterstyle{jrmbsc}

% Sæt numre på subsections (men put dem ikke i indholdsfortegnelsen)
\setsecnumdepth{subsection}


% ~~~~~~~~~~~~~~~~~~~~~~~~~~~~~~~~~~~~~~~~~~~~~~~~~~~~~~~~~~~~~~~~~~~~~~~~~~~~~
% Flydende ting
% ~~~~~~~~~~~~~~~~~~~~~~~~~~~~~~~~~~~~~~~~~~~~~~~~~~~~~~~~~~~~~~~~~~~~~~~~~~~~~
% Include graphics
% \usepackage{graphicx}  % ALREADY IMPORTED

% Subfloats
\newsubfloat{figure}
\subcaptionstyle{\raggedright}

% Brug sans-serif til figurtekster
\captionnamefont{\sffamily\scshape}
\captiontitlefont{\sffamily\small}

% Pæne fodnoter (memoir tricks)
\setlength{\footmarkwidth}{-1sp}
\setlength{\footmarksep}{0em}
\footmarkstyle{#1: }


% ~~~~~~~~~~~~~~~~~~~~~~~~~~~~~~~~~~~~~~~~~~~~~~~~~~~~~~~~~~~~~~~~~~~~~~~~~~~~~
% Science
% ~~~~~~~~~~~~~~~~~~~~~~~~~~~~~~~~~~~~~~~~~~~~~~~~~~~~~~~~~~~~~~~~~~~~~~~~~~~~~
% Basal matematik
\usepackage{amsmath}

% Bad-ass math!
\usepackage{mathtools}
\mathtoolsset{showonlyrefs=false,showmanualtags}

% Enheder
\usepackage{siunitx}
\sisetup{separate-uncertainty=true}

% Sådan laver man sine egne enheder
\DeclareSIUnit\au{AU}
\DeclareSIUnit\year{yr}
\DeclareSIUnit\erg{erg}
\DeclareSIUnit\msun{M_{\odot}}
\DeclareSIUnit\lsun{L_{\odot}}
\DeclareSIUnit\byte{B}


% ~~~~~~~~~~~~~~~~~~~~~~~~~~~~~~~~~~~~~~~~~~~~~~~~~~~~~~~~~~~~~~~~~~~~~~~~~~~~~
% Stuff
% ~~~~~~~~~~~~~~~~~~~~~~~~~~~~~~~~~~~~~~~~~~~~~~~~~~~~~~~~~~~~~~~~~~~~~~~~~~~~~
% Automatisk afstand i makroer
\usepackage{xspace}

% Indstilling af lister
\usepackage{enumitem}

% Faver
\usepackage{xcolor}

% Vrøvletekst
\usepackage{lipsum}

% ~~~~~~~~~~~~~~~~~~~~~~~~~~~~~~~~~~~~~~~~~~~~~~~~~~~~~~~~~~~~~~~~~~~~~~~~~~~~~
% Henvisninger
% ~~~~~~~~~~~~~~~~~~~~~~~~~~~~~~~~~~~~~~~~~~~~~~~~~~~~~~~~~~~~~~~~~~~~~~~~~~~~~
% Smarte anførselstegn (kræves også af eksempelvis biblatex)
\usepackage{csquotes}

% Skriv URL'er
\usepackage{url}

% Bibliografi med biblatex og biber
\usepackage[backend=biber,
  style=numeric,
  uniquelist=false,
  uniquename=false,
  doi=false,
  url=false,
  isbn=false,
  eprint=false,
  hyperref=true]{biblatex}

% Indlæs bibliografi-filen med det bestemte format
\addbibresource{bibliography.bib}

% Afstand mellem kilderne
\setlength\bibitemsep{1.1\itemsep}

% DANISH STUFF: Skriv et al. på dansk (i stedet for 'm.fl.')
\DefineBibliographyStrings{danish}{%
  andothers = {et\addabbrvspace al\adddot}
}

% DANISH STUFF: Titlen på bibliografien
\defbibheading{memoirbib}[Litteraturliste]{%
  \chapter*{#1} \addcontentsline{toc}{chapter}{#1}}

% Henvisningspakker (rækkefølgen er vigtig!)
% --> Til henvisninger, brug: \cref{}  or \Cref{} !
\usepackage{refcount}
\usepackage{varioref}
\usepackage[
  unicode=true,
  pdftitle={\projecttitle},
  pdfauthor={\theauthor},  % Change this name?
  pdfkeywords={},
  bookmarksopen=true,
  pdfdisplaydoctitle=true,
  colorlinks=false,
  pdfborder={0 0 0},
  hypertexnames=false]{hyperref}  % Gør links klikbare
\usepackage{cleveref}


% ~~~~~~~~~~~~~~~~~~~~~~~~~~~~~~~~~~~~~~~~~~~~~~~~~~~~~~~~~~~~~~~~~~~~~~~~~~~~~
% Makroer / kommandoer (eksempler på hvordan det kan gøres)
% ~~~~~~~~~~~~~~~~~~~~~~~~~~~~~~~~~~~~~~~~~~~~~~~~~~~~~~~~~~~~~~~~~~~~~~~~~~~~~
% Lille opdeling (blank linje med stjerner)
\newcommand\starbreak{\medskip\fancybreak{$*$\qquad$*$\qquad$*$}\medskip}

% Pæn subscript
\newcommand\var[2]{\ensuremath{#1_{\textup{#2}}\,}\xspace}

% Pæn afstand med komma i en ligning
\newcommand\eqsep{\ensuremath{\quad , \quad}}

% Definition af navne
\newcommand\numpy{NumPy\xspace}
\newcommand\scipy{SciPy\xspace}
\newcommand\kepler{\textsl{Kepler}\xspace}

% Afledte
% --> Overvej at tage et kig på pakke 'derivative' i stedet
\newcommand\diff[2]{\frac{\text{d}#1}{\text{d}#2}}
\newcommand\pdiff[2]{\frac{\partial #1}{\partial #2}}
\newcommand\idiff[2]{\ensuremath{\text{d}#1/\text{d}#2}}

% Ting der ofte genbruges
\newcommand\teff{\ensuremath{T_{\textup{eff}}}\xspace}
\newcommand\logg{\ensuremath{\log g}\xspace}